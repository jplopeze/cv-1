\documentclass[a4paper,plain]{memoir}
\usepackage[ignoreall]{geometry}
\usepackage{helvet,xifthen,tikz}
\usepackage{xspace,graphicx, xcolor}
\usetikzlibrary{calc,arrows,positioning,fit}


\usepackage{hyperref}
\hypersetup{colorlinks,urlcolor=black}
\newcommand{\Href}[2]{\href{#1}{\em #2}}


\pagestyle{empty}

\definecolor{uucol}{RGB}{191,31,50}
\colorlet{seccol}{uucol!70!black}

\input{tikzstyles.tex}

\colorlet{colBI}{darkC}
\colorlet{colUP}{darkD}
\colorlet{colSZG}{darkB}

\colorlet{colwork}{colorA}
\colorlet{coleducation}{colorE}

\tikzstyle move=[very thick, shorten <= 0mm, shorten >= 1mm, ->]
\tikzstyle timeline=[white]%[very thick, ->]
\tikzstyle movelabel=[sloped,above,fill=white,text=seccol]
\begin{document}

%%% BACKGROUND MAP
\begin{tikzpicture}[overlay,remember picture,scale=1.5]
  \node[anchor=south east] at (current page.south east) (Map) {\includegraphics[scale=1.5,trim=310 200 300 60,clip]{europe}};

  % block out england
  \draw (current page.south east) +(145:9.5cm) node[fill=white,rotate=80,minimum height=2.2cm, minimum width=8cm] {};


  \draw (Map) ++(-94:4.4cm) node[circle, fill=colSZG,text=white, minimum size=1cm] (SZG) {Salzburg};
  \draw (Map) ++(235:2.7cm) node[circle, fill=colBI,text=white, minimum size=1cm] (BI) {Bielefeld};
  \draw (Map) ++(40:1.2cm) node[circle, fill=colUP,text=white, minimum size=1cm] (UP) {Uppsala};

  \draw[move] (SZG) to (BI);
  \draw[move] (BI) to (UP);
\end{tikzpicture}

\def\maxyear{2016}
\def\maxyearplusone{2017}
%%% BACKGROUND TIME
\begin{tikzpicture}[overlay, remember picture]
  \coordinate (Start) at ($ (current page.south west) + (1.5cm,0cm)$);
  \coordinate (End) at ($ (current page.north west) + (1.5cm,-5cm)$);

  \draw[timeline] (Start) -- (End);

  \foreach \year in {2006,...,\maxyearplusone} {
    \pgfmathparse{\maxyear-2006}
    \coordinate (Y\year M1) at ($(End)!\maxyear/\pgfmathresult - \year/\pgfmathresult!(Start)$);
    \coordinate (Y\year) at ($(End)!\maxyear/\pgfmathresult - \year/\pgfmathresult!(Start)$);
    \node[anchor=center, xshift=-5mm] at (Y\year) {\year};
%    \foreach \month in {1,...,12} {
%      \draw (Y\year) ++(90:0.08333cm*\month)
%    }
  }

  \foreach \year in {2006,...,\maxyear} {
    \foreach \month in {2,...,12} {
      \pgfmathparse{\year+1}
      \coordinate (nextyear) at (Y\pgfmathresult);
      \coordinate (Y\year M\month) at ($ (Y\year)!\month/12-1/12!(nextyear) $);
      
      \ifthenelse{6 = \month}{
        \draw (Y\year M\month) [xshift=-5mm] ++(-1.0mm,0) -- +(2mm,0);
      }{
        \draw (Y\year M\month) [xshift=-5mm] ++(-0.5mm,0) -- +(1mm,0);
      }
    }
  }

  %% hide last year...
%  \node[fill=white,minimum size=2cm] at (Y\maxyear M1) {};
  \node[fill=white,minimum size=2cm] at (Y2006M1) {};
\end{tikzpicture}

% parameters: #1:title, #2:starttime, #3:endtime, #4:place, #5:description, #6:offset, #7:pos, #8:anchor
\newcommand{\opentimespan}[8]{
  \begin{tikzpicture}[overlay,remember picture]
    \coordinate (linestart) at ($ #2 + (0:0.5cm) + (0:#6) $);
    \coordinate (lineend) at ($ #3 + (0:0.5cm) + (0:#6) $);
    \node [anchor=#8, text width=7.5cm,fill=white,xshift=2mm] at ($ (lineend)!#7!(linestart) $) {{\em\textcolor{col#4}{#1}} -- #5};
    \draw[|->,color=col#4, line width=1pt] (linestart) -- (lineend);
  \end{tikzpicture}
}


% parameters: #1:title, #2:starttime, #3:endtime, #4:place, #5:description, #6:offset, #7:pos, #8:anchor
\newcommand{\timespan}[8]{
  \begin{tikzpicture}[overlay,remember picture]
    \coordinate (linestart) at ($ #2 + (0:0.5cm) + (0:#6) $);
    \coordinate (lineend) at ($ #3 + (0:0.5cm) + (0:#6) $);
    \node [anchor=#8, text width=7.5cm,fill=white,xshift=2mm] at ($ (lineend)!#7!(linestart) $) {{\em\textcolor{col#4}{#1}} -- #5};
    \draw[|-|,color=col#4, line width=1pt] (linestart) -- (lineend);
  \end{tikzpicture}
}

% parameters: #1:title, #2:time, #3:place, #4:description, #5:offset
\newcommand{\event}[5]{
  \timespan{#1}{#2}{#2}{#3}{#4}{#5}{0}{west}
}

% \newcommand{\tagtwo}[2]{\textcolor{col#1}{\texttt{\##2}}}
\newcommand{\tagtwo}[2]{}
\newcommand{\tag}[1]{\tagtwo{#1}{#1}}

%%%%%%%%%%%%%%%% DATA

\opentimespan{Ph.D student, Uppsala
  University}{(Y2013M4)}{(Y2016M10)}{UP}{\\Developing type systems
  for alias control; supervised by Tobias Wrigstad and Dave
  Clarke. \tag{work} \tag{education}}{0cm}{0.825}{north west}

\event{Publication $\times$
  2}{(Y2015M10)}{UP}{\Href{http://stbr.me/Disjointness-Domains-for-Fine-Grained-Aliasing}{``Disjointness
    Domains for Fine-Grained Aliasing''}, Brandauer, Wrigstad, Object-Oriented
  Programming, Systems,
  Languages \& Applications.\\
  \Href{http://stbr.me/Encore-Glimpse}{``Parallel Objects for Multicores: A
    Glimpse at the Parallel Language Encore''}, \small{Brandauer, Castegren, Clarke,
  Fernández, Broch Johnsen, Pun, Tapia Tarifa, Wrigstad, Yang, International
  School on Formal Methods for the Design of Computer, Communication and
  Software Systems: Multicore Programming}}{0.5cm}


\event{Publication}{(Y2012M6)}{UP}{\Href{http://lame.dei.uc.pt/images/1/11/Lame12_submission_3.pdf}{\\``The
    Joelle Programming Language''}, \small{\"{O}stlund, Brandauer,
  Wrigstad, International Workshop on Languages for the Multi-Core
  Era, ECOOP'12.} \tag{education}}{0.5cm}

\timespan{M.Sc. in CS, Uppsala
  University}{(Y2010M9)}{(Y2013M3)}{UP}{M.Sc. thesis ``Task
  Scheduling using Joelle's Effects''. Implementing a task
  scheduler for a parallel programming language.
  \tag{education}}{0cm}{0.5}{north west}

\timespan{B.Sc. in Cognitive
  Informatics}{(Y2007M10)}{(Y2010M8)}{BI}{Average grade 1.5 (grades
  1-5, 1 best), B.Sc. thesis: 2D physics engine \tag{education}}{0cm}{0.6}{north
  west}

\timespan{Freelancing at Comet Consulting} {(Y2006M9)}
         {(Y2008M3)}{SZG}{Developing measuring software in C\# for
           automatic 3D laser-range-scan data on construction
           sites \tag{work}}{0.5cm}{0}{north west}

\timespan{Teaching- and Research Assistant}{(Y2009M2)}{(Y2010M6)}{BI}{Teach AI,
  develop VR and eye tracking apps. \tag{work} \tag{education}}{0.5cm}{0}{north west}

\event{Publication}{(Y2009M8)}{BI}{\Href{http://www.techfak.uni-bielefeld.de/ags/wbski/hiwis/sbrandau/files/navigation_in_virtual_reality_with_the_wii_balance_board.pdf}{``Navigation in VR with the Wii Balance Board''}, \small{Hilsendeger, Brandauer, Tolksdorf,
  Fr\"{o}hlich, 6th Workshop on VR/AR,
  2009 \tag{education}}}{1cm}

\timespan{Laube GmbH}{(Y2006M9)}{(Y2007M6)}{SZG}{Social work instead of being
  drafted for military service. \tag{work}}{1cm}{0.5}{west}

%%%%%%%%%%%%%%%%% PERSONAL DETAILS
\begin{tikzpicture}[overlay,remember picture]
%  \draw (current page.north east) ++(-135:2cm) rectangle +(-6,-10);
%  \draw (current page.north east) ++(-135:2cm) node [minimum width=7cm, minimum height=8cm,draw, anchor=north east] (Box) {};
  
  \draw (current page.north east) ++(-135:3cm) node
        [rotate=0,anchor=south east,inner sep=1pt] (Name) {\huge
          \textsc{Stephan Brandauer}};

        \draw (Name.west) ++(0,-1em) node[text width=5.8cm,anchor=north
        west,inner sep=1pt] (Description)
        {\begin{flushleft}\href{http://stbr.me/work}{\texttt{http://stbr.me/work}}\\Born
            and raised near Salzburg, Austria, I studied in Germany and Sweden.
            I work, teach, and learn as a Ph.D. student at Uppsala University.
            I'm interested in understanding how mutable state is used by
            practitioners, and in creating language abstractions that let
            practitioners control aliasing of mutable state. This serves two
            interests: writing correct software, and writing efficient software.
          \end{flushleft}};

  \draw (Name.south east) ++(0:2mm) node[rotate=90,inner
    sep=1pt,anchor=south east] (Contact) {+46 700 599 236,
    stephan.brandauer@it.uu.se, @sbrandauer};
\end{tikzpicture}

\end{document}
