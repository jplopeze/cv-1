\documentclass[a4paper,plain]{memoir}
\usepackage[ignoreall]{geometry}
\usepackage{helvet,xifthen,tikz}
\usepackage{xspace,graphicx, xcolor}
\usetikzlibrary{calc,arrows,positioning,fit}


\usepackage{hyperref}
\hypersetup{colorlinks,urlcolor=black}
\newcommand{\Href}[2]{\href{#1}{\em #2}}

\pagestyle{empty}

\definecolor{uucol}{RGB}{191,31,50}
\colorlet{seccol}{uucol!70!black}

\input{tikzstyles.tex}

\colorlet{colBI}{darkC}
\colorlet{colUP}{darkD}
\colorlet{colSZG}{darkB}

\colorlet{colwork}{colorA}
\colorlet{coleducation}{colorE}

\tikzstyle move=[very thick, shorten <= 0mm, shorten >= 1mm, ->]
\tikzstyle timeline=[white]%[very thick, ->]
\tikzstyle movelabel=[sloped,above,fill=white,text=seccol]
\begin{document}

%%%%%% BACKGROUND MAP
\begin{tikzpicture}[overlay,remember picture,scale=1.5]
  \node[anchor=south east] at (current page.south east) (Map) {\includegraphics[scale=0.8,trim=310 200 300 60,clip]{europe}};

  % block out england
  \draw (current page.south east) +(145:5.5cm) node[fill=white,rotate=70,minimum height=2.2cm, minimum width=8cm] {};


  \draw (Map) ++(-94:2.3cm) node[circle, fill=colSZG,text=white, minimum size=1cm] (SZG) {\tiny Salzburg};
  \draw (Map) ++(240:1.5cm) node[circle, fill=colBI,text=white, minimum size=1cm] (BI) {\tiny Bielefeld};
  \draw (Map) ++(35:0.7cm) node[circle, fill=colUP,text=white, minimum size=1cm] (UP) {\tiny Uppsala};

  \draw[move] (SZG) to (BI);
  \draw[move] (BI) to (UP);
\end{tikzpicture}

\def\maxyear{2016}
\def\maxyearplusone{2017}
%%% BACKGROUND TIME
\begin{tikzpicture}[overlay, remember picture]
  \coordinate (Start) at ($ (current page.south west) + (1.5cm,0cm)$);
  \coordinate (End) at ($ (current page.north west) + (1.5cm,-5cm)$);

  \draw[timeline] (Start) -- (End);

  \foreach \year in {2006,...,\maxyearplusone} {
    \pgfmathparse{\maxyear-2006}
    \coordinate (Y\year M1) at ($(End)!\maxyear/\pgfmathresult - \year/\pgfmathresult!(Start)$);
    \coordinate (Y\year) at ($(End)!\maxyear/\pgfmathresult - \year/\pgfmathresult!(Start)$);
    \node[anchor=center, xshift=-5mm] at (Y\year) {\year};
%    \foreach \month in {1,...,12} {
%      \draw (Y\year) ++(90:0.08333cm*\month)
%    }
  }

  \foreach \year in {2006,...,\maxyear} {
    \foreach \month in {2,...,12} {
      \pgfmathparse{\year+1}
      \coordinate (nextyear) at (Y\pgfmathresult);
      \coordinate (Y\year M\month) at ($ (Y\year)!\month/12-1/12!(nextyear) $);
      
      \ifthenelse{6 = \month}{
        \draw (Y\year M\month) [xshift=-5mm] ++(-1.0mm,0) -- +(2mm,0);
      }{
        \draw (Y\year M\month) [xshift=-5mm] ++(-0.5mm,0) -- +(1mm,0);
      }
    }
  }

  %% hide last year...
%  \node[fill=white,minimum size=2cm] at (Y\maxyear M1) {};
  \node[fill=white,minimum size=2cm] at (Y2006M1) {};
\end{tikzpicture}

% parameters: #1:title, #2:starttime, #3:endtime, #4:place, #5:description, #6:offset, #7:pos, #8:anchor
\newcommand{\opentimespan}[8]{
  \begin{tikzpicture}[overlay,remember picture]
    \coordinate (linestart) at ($ #2 + (0:0.5cm) + (0:#6) $);
    \coordinate (lineend) at ($ #3 + (0:0.5cm) + (0:#6) $);
    \node [anchor=#8, text width=7.5cm,fill=white,xshift=2mm] at ($ (lineend)!#7!(linestart) $) {{\em\textcolor{col#4}{#1}} -- #5};
    \draw[|->,color=col#4, line width=1pt] (linestart) -- (lineend);
  \end{tikzpicture}
}


% parameters: #1:title, #2:starttime, #3:endtime, #4:place, #5:description, #6:offset, #7:pos, #8:anchor
\newcommand{\timespan}[8]{
  \begin{tikzpicture}[overlay,remember picture]
    \coordinate (linestart) at ($ #2 + (0:0.5cm) + (0:#6) $);
    \coordinate (lineend) at ($ #3 + (0:0.5cm) + (0:#6) $);
    \node [anchor=#8, text width=7.5cm,fill=white,xshift=2mm] at ($ (lineend)!#7!(linestart) $) {{\em\textcolor{col#4}{#1}} -- #5};
    \draw[|-|,color=col#4, line width=1pt] (linestart) -- (lineend);
  \end{tikzpicture}
}

% parameters: #1:title, #2:time, #3:place, #4:description, #5:offset
\newcommand{\event}[5]{
  \timespan{#1}{#2}{#2}{#3}{#4}{#5}{0}{west}
}

%%%%%%%%%%%%%%%% DATA

\opentimespan{Ph.D student, Uppsala
  University}{(Y2013M4)}{(Y2016M10)}{UP}{\\\textbf{Developing a type system} for
  alias control, working on research programming language \textbf{compiler and
    runtime}, and implementing a \textbf{dynamic~analysis tool for Java byte
    code} -- and \textbf{analyse data using Cassandra and Spark}; supervised by
  Tobias Wrigstad and Dave Clarke.}{0cm}{0.625}{north west}

\event{2 Publications}{(Y2015M10)}{UP}{\Href{http://stbr.me/Disjointness-Domains-for-Fine-Grained-Aliasing}{``Disjointness
    Domains for Fine-Grained Aliasing''}, Brandauer, Wrigstad, Object-Oriented
  Programming, Systems,
  Languages \& Applications.\\
  \Href{http://stbr.me/Encore-Glimpse}{``Parallel Objects for Multicores: A
    Glimpse at the Parallel Language Encore''}, \small{Brandauer, Castegren, Clarke,
  Fernández, Broch Johnsen, Pun, Tapia Tarifa, Wrigstad, Yang, International
  School on Formal Methods for the Design of Computer, Communication and
  Software Systems: Multicore Programming.}}{0.5cm}


\event{Publication}{(Y2012M6)}{UP}{\Href{http://lame.dei.uc.pt/images/1/11/Lame12_submission_3.pdf}{\\``The
    Joelle Programming Language''}, \small{\"{O}stlund, Brandauer,
  Wrigstad, International Workshop on Languages for the Multi-Core
  Era, ECOOP'12.}}{0.5cm}

\timespan{M.Sc. in CS, Uppsala
  University}{(Y2010M9)}{(Y2013M3)}{UP}{M.Sc. thesis ``Task
  Scheduling using Joelle's Effects''. Implementing a task
  scheduler for a parallel programming language.}{0cm}{0.5}{north west}

\timespan{B.Sc. in Cognitive Informatics, Bielefeld
  University}{(Y2007M10)}{(Y2010M8)}{BI}{Average grade 1.5 (grades 1-5, 1 best),
  B.Sc. thesis: 2D physics engine.}{0cm}{0.6}{north west}

\timespan{Freelancing at Comet Consulting} {(Y2006M9)}
         {(Y2008M3)}{SZG}{Developing measuring software in C\# for
           automatic 3D laser-range-scan data on construction
           sites.}{0.5cm}{0}{north west}

\timespan{Teaching- and Research Assistant}{(Y2009M2)}{(Y2010M6)}{BI}{Teach AI,
  develop VR and eye tracking apps.}{0.5cm}{0}{north west}

\event{Publication}{(Y2009M8)}{BI}{\Href{http://www.techfak.uni-bielefeld.de/ags/wbski/hiwis/sbrandau/files/navigation_in_virtual_reality_with_the_wii_balance_board.pdf}{``Navigation in VR with the Wii Balance Board''}, \small{Hilsendeger, Brandauer, Tolksdorf,
  Fr\"{o}hlich, 6th Workshop on VR/AR,
  2009.}}{1cm}

\timespan{Laube GmbH}{(Y2006M9)}{(Y2007M6)}{SZG}{Social work instead of being
  drafted for military service.}{1cm}{0.5}{west}

%%%%%%%%%%%%%%%%% PERSONAL DETAILS
\begin{tikzpicture}[overlay,remember picture]
%  \draw (current page.north east) ++(-135:2cm) rectangle +(-6,-10);
%  \draw (current page.north east) ++(-135:2cm) node [minimum width=7cm, minimum height=8cm,draw, anchor=north east] (Box) {};
  
  %\draw (current page.north east) ++(-135:3cm) node
  \draw (current page.north east) ++(-155:3cm) node
        [rotate=0,anchor=south east,inner sep=1pt] (Name) {\hspace{1mm}\huge
          \textsc{Stephan Brandauer}};

        \draw (Name.west) ++(0,-1em) node[text width=6.9cm,anchor=north
        west,inner sep=1pt] (Description)
        {

            \hspace{-3mm}\begin{tabular}{rl}
            Web & \url{http://stbr.me/work}\\
            Mail & \href{mailto:stephan.brandauer@it.uu.se}{\texttt{stephan.brandauer@it.uu.se}}\\
            Github & \url{https://github.com/kaeluka}\\
            LinkedIn & \href{http://www.linkedin.com/in/stephan-brandauer}{\texttt{http://www.linkedin.com/in/}}\\
            & \href{http://www.linkedin.com/in/stephan-brandauer}{\texttt{stephan-brandauer}}\\
            Phone & +46 700 599 236\\
            \end{tabular}\vspace{3mm}

            Born and raised near Salzburg, Austria, I studied in Germany and
            Sweden. I'm currently a \textbf{Ph.D.~student} at Uppsala University.

            \vspace{2mm}My research is about understanding how \textbf{shared
              mutable state} is used in practise, and about creating
            abstractions that let practitioners control sharing of mutable
            state. This serves two interests: writing correct software, and
            writing efficient software.

            \vspace{2mm}As a programmer, I am looking to learn how to develop
            \textbf{large scale distributed systems} with \textbf{tight
              performance constraints}. I also love all things related to
            information visualisation and would love to learn more about static
            or dynamic code analysis in practise.
            
            \vspace{2mm}I have coded in many languages, some recent are \textbf{C++,
            Java, Scala, Haskell}. I like Haskell for its elegance, I like Java
            for its pragmatism, I like Scala for being a little bit of both. I
            don't want to like C++, but I also don't know how to quit.
          };

\end{tikzpicture}

\end{document}
